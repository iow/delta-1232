\documentclass[12pt]{revtex4-1}
\DeclareRobustCommand{\baselinestretch{1.3}}

\usepackage{amssymb,amsthm,amsmath,amsbsy,color,paralist}


\newcommand{\jx}{\left( \begin{array}{ccc}
0 & 0 & 0 \\
0 & 0 & -i \\
0 & i & 0 \\
\end{array}
\right)}

\newcommand{\jy}{\left( \begin{array}{ccc}
0 & 0 & i \\
0 & 0 & 0 \\
-i & 0 & 0 \\
\end{array}
\right)}

\newcommand{\jz}{\left( \begin{array}{ccc}
0 & -i & 0 \\
i & 0 & 0 \\
0 & 0 & 0 \\
\end{array}
\right)}

\begin{document}
\section{Massive Spin $S=1$ Polarization Vectors}

For massive particle with spin 1 there are three independent polarizations.
They are described by polarization vector $\epsilon^{\mu}(\vec{p}, \lambda)$,
where $\lambda = 0$ stands for longitudinal polarization (absent in case
of massless particle), $\lambda = \pm 1$ are right and left polarized 
respectively.
We will talk about polarization along  momentum $\vec{p}$, i.e. \textit{helicity}.
Such convention is meaningless in the rest frame (for a massive particle),
but in such case all three polarizations are equivalent. Anyway, we will
mostly work in center of mass frame (with other particle present), so 
below equations will be essential.

General formula for polarization 4-vector can be written as follows:
\begin{equation}
	\epsilon^{\mu}(\vec p, \lambda) = \left( \frac{\vec{p}\cdot 
	\vec{\varepsilon_{\lambda}}}{M} , \vec{\varepsilon_{\lambda}}
	+ \frac{\vec{p}\cdot 
	\vec{\varepsilon_{\lambda}}}{p^0 + M} \frac{\vec{p}}{M}
	\right),
\end{equation}
with $\vec{\varepsilon}_{\lambda}$ - some 3-vectors, which also called
polarization vectors ($p^0 = \sqrt{\vec{p}^2 + M^2}$).
For particle with momentum along $z$-axis, this vectors are straightforward:
\begin{align} \label{eqn:polarization-3-vectors}
	\vec{\varepsilon}_{\lambda = 0}= & (0, 0 ,1), \\
	\vec{\varepsilon}_{\lambda = \pm 1}= & \mp \frac{1}{\sqrt{2}}(1, \pm i,0).
\end{align}
For general $\vec{p} = (p \sin\theta \cos\phi, p \sin\theta \sin\phi,
p \cos(\phi)$ corresponding expressions are bit cumbersome.

For \textbf{longitudinal} polarization, polarization is collinear with
monentum vector, i.e. $\vec{\varepsilon}_{\lambda = 0} = \vec{p}/|\vec{p}|$.
Substitution to general formula yields:
\begin{equation}
	\epsilon^{\mu}(\vec p, \lambda = 0) = \left( \frac{|\vec{p}|}{M} ,
	\frac{p^0 \vec{p}}{M |\vec{p}|}	\right).
\end{equation}
To obtain \textbf{transversal} polarization, we will use another trick.
Suppose we work in coordinate system, where $\vec{p} = p \hat{z}$.
Then in this frame polarization 3-vectors are taken from 
(\ref{eqn:polarization-3-vectors}). Now we apply rotational operators
to this vectors, to get spinors, which satisfy following condition:
\begin{equation}
	(\vec{J} \cdot \vec{n} ) \epsilon_{\lambda} = \lambda \epsilon_{\lambda}.
\end{equation}
Here $\vec{J} = (J_1, J_2, J_3)$ - spin operator, which in our representation
has the form:
\begin{equation}
	J_1 = \jx,\, J_2 = \jy,\, J_3 = \jz. 
\end{equation}
Then, we write (quite general) rotational operator:
\begin{equation}
	R(\theta, \phi) = e^{-i \phi J_3} e^{- i \theta J_2} e^{+ i \phi J_3}.
\end{equation}
Taking matrix exponent is straightforward:
\begin{equation*}
e^{-i \phi J_3} = 
\left(
\begin{array}{ccc}
 \cos \phi & -\sin\phi & 0 \\
 \sin \phi & \cos \phi & 0 \\
 0 & 0 & 1 \\
\end{array}
\right),\,
e^{-i \theta J_2} = 
\left(
\begin{array}{ccc}
 \cos \theta & 0 & \sin \theta \\
 0 & 1 & 0 \\
 -\sin \theta & 0 & \cos \theta \\
\end{array}
\right).
\end{equation*}
Using rotational operator on $\vec \varepsilon_{\lambda = 0}$ gives us
previous result. For $\vec \varepsilon_{\lambda = \pm 1}$ one can get
\begin{align} \label{eqn:polarization-3-vectors}
	\vec{\varepsilon}_{\lambda = +1}= & -\frac{1}{\sqrt{2}} &
	\left(\cos^2 \frac{\theta}{2} - e^{+2 i \phi}\sin^2 \frac{\theta}{2} ,
	+i(\cos^2 \frac{\theta}{2} + e^{+2 i \phi}\sin^2 \frac{\theta}{2}) ,
	- e^{+i \phi} sin \theta \right),
	\\
	\vec{\varepsilon}_{\lambda = -1}= & +\frac{1}{\sqrt{2}} &
	\left(\cos^2 \frac{\theta}{2} - e^{-2 i \phi}\sin^2 \frac{\theta}{2} , 
	-i(\cos^2 \frac{\theta}{2} + e^{-2 i \phi}\sin^2 \frac{\theta}{2}) ,
	- e^{-i \phi} sin \theta \right),
\end{align}
Substituting these 3-vectors to general formula for 4-vector, one gets:
\begin{equation}
	\epsilon^{\mu}(\vec p, \lambda = \pm 1) = \left( 0 ,
	\vec{\varepsilon}_{\lambda = \pm 1} \right).
\end{equation}

Notice, that $p_{\mu} \epsilon^{\mu} = 0$ for $p$ on mass shell.

\end{document}