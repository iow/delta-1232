\documentclass[12pt]{revtex4-1}
\DeclareRobustCommand{\baselinestretch{1.3}}


\usepackage{feynmp}
\usepackage{amsmath,amssymb,amsthm,amsbsy,color,paralist}
\usepackage[titletoc]{appendix}

\newcommand{\hp}{{\frac{1}{2}}}
\newcommand{\hm}{{-\frac{1}{2}}}
\newcommand{\tp}{{\frac{3}{2}}}
\newcommand{\tm}{{-\frac{3}{2}}}

\newcommand{\bk}{{\boldsymbol k}}
\newcommand{\bq}{{\boldsymbol q}}
\newcommand{\bp}{{\boldsymbol p}}
\newcommand{\bK}{{\boldsymbol K}}
\newcommand{\bP}{{\boldsymbol P}}

\newcommand{\cG}{{\cal G}}
\newcommand{\cF}{{\cal F}}
\newcommand{\cma}{{\theta_{cm}}}
\newcommand{\cM}{{\cal M}}

\newcommand{\jx}{\left( \begin{array}{ccc}
0 & 0 & 0 \\
0 & 0 & -i \\
0 & i & 0 \\
\end{array}
\right)}
\newcommand{\jy}{\left( \begin{array}{ccc}
0 & 0 & i \\
0 & 0 & 0 \\
-i & 0 & 0 \\
\end{array}
\right)}
\newcommand{\jz}{\left( \begin{array}{ccc}
0 & -i & 0 \\
i & 0 & 0 \\
0 & 0 & 0 \\
\end{array}
\right)}


\begin{document}
	\unitlength = 1mm
	% determine the unit for the size of diagram.
\section{ $eN \to e\Delta$ reaction}
	Under certain circumstances, reaction $eN \to e\Delta$ reaction
	can occur. In lowest order, amplitude can be discribed by the
	tree-level diagram (fig. \ref{fig:delta-amp}).
	\begin{figure} \centering \label{fig:delta-amp}
	\begin{fmffile}{delta-amp}
		\begin{fmfgraph*}(40,20)
			\fmfbottom{i1,d1,o1}
			\fmftop{i2,d2,o2}
			\fmf{double,label=$\Delta(\mu,,\tau)$}{v1,o1}
			\fmfv{decor.shape=circle,decor.filled=hatched,
				decor.size=3thick}{v1}
			\fmf{fermion,label=$N(p,,\lambda)$}{i1,v1}
			\fmf{fermion,label=$e(k,,h)$}{i2,v2}
			\fmf{fermion,label=$e(k',,h')$}{v2,o2}
			\fmf{photon,tension=0,label=$\gamma$}{v1,v2}
		\end{fmfgraph*}
	\end{fmffile}
	\caption{Tree-level contribution to reaction $eN \to e\Delta$.}
	\end{figure}
	
	Analytically this amplitude can be expressed in terms of photon propagator,
	fermionic spinors and vertices:
	\begin{equation}
		T = \frac{e^2}{-q^2} \bar{u}(k', h') \gamma_\mu u(k, h) \cdot
		\bar{\psi}_\alpha (p', \tau) \Gamma^{\alpha \mu} u(p, \lambda),
	\end{equation}
	where $\psi_\alpha (p', \tau)$ - spin $3/2$ particle spinor for $\Delta$
	(also known as Rarita-Schwinger spinor). $\gamma N \Delta$ vertice there
	can be described as following:
	
	\begin{align} \label{eqn:big-gamma}
	\Gamma^{\alpha \mu} = \sqrt{\frac{2}{3}}& \{ (\gamma^\mu 
	q^\alpha - \hat{q} g^{\alpha \mu})g_1(Q^2) 
	\\
	& + (q \cdot p' g^{\alpha \mu} - q^\alpha p^{\prime \mu}) g_2(Q^2)
	\\
	& + (q^\alpha q^\mu - q^2 g^{\alpha \mu}) g_3(Q^2)\} i \gamma_5,
	\end{align}

	\begin{equation}
		(\hat{p} - M_\Delta) \psi_\alpha = 0, \,\,\,\, \alpha = 0,1,2,3
	\end{equation}
	
	\begin{equation}
		p_\alpha \psi^\alpha(p, \tau) = 0.
	\end{equation}

	For different incoming and outcoming particles, one can construct
	$2 \times 2 \times 2 \times 4 = 32$ amplitudes $T_{h' \tau, h \lambda}$.
	However, in a limit of zero electron mass amplitudes with lepton 
	spin flip are zero. Moreover, parity conservation suggests following
	relation \cite{bib:leader}:
	\begin{equation*}
		T_{h' \tau, h \lambda}(\theta, \phi) 
		= \frac{\eta_e \eta_p}{\eta_e \eta_\Delta}
		(-1)^{J_e + J_p - J_e - J_\Delta} (-1)^{(h - \lambda) - (h' - \tau)}
		T_{-h' -\tau, -h -\lambda} (\theta,-\phi) ,
	\end{equation*}
	which in our case takes the form
	\begin{equation}
		T_{h' \tau, h \lambda}(\theta, \phi) 
		=
		(-1) (-1)^{(h - \lambda) - (h' - \tau)}
		T_{-h' -\tau, -h -\lambda} (\theta,-\phi) 
	\end{equation}
	Therefore, only $8$ independent amplitudes should be considered.
	
	One can calculate those amplitudes directly substituting helicity spinors
	and taking products. By convention, we use following spinors:
	\begin{equation}
		u_e(\vec k, h) = \sqrt{k_0 + m_e} \left(
			\begin{array}{ll}
				\chi_h(\theta, \phi) \\
				\frac{\vec{\sigma} \vec{k}}{k_0 + m_e} \chi_h(\theta, \phi)
			\end{array}
			\right),
	\end{equation}
	where $\chi$ is helicity two-spinors. For electron, we take:
	\begin{equation}
		\chi_{h=\hp} = 
		\left( \begin{array}{c}
		cos \frac{\cma}{2} \\
		e^{i \phi} sin \frac{\cma}{2} 
		\end{array} \right) ~~~~
		\chi_{h=\hm} =  
		\left( \begin{array}{c}
		- e^{- i \phi}  sin \frac{\cma}{2} \\
		cos \frac{\cma}{2} 
		\end{array} \right).  \nonumber
	\end{equation}
	When we write down spinor for countering nucleon, by Jacob-Wick convention
	we take opposite spinors for $\chi_{\lambda=\hm}$ and $\chi_{\lambda=\hp}$.
	Additionaly, one should not forget about non-vanishing nucleon mass $M_N$.
	For Rarita-Schwinger helicity spinor, we use following representation:
	\begin{align}
		 \psi^{\mu} (\vec p, +\frac{3}{2}) = &
		 \varepsilon^{\mu}(\vec p, +1) u(\vec p, \hp),
		 \\
		 \psi^{\mu} (\vec p, +\frac{1}{2}) = &
		 \sqrt{\frac{2}{3}} \varepsilon^{\mu}(\vec p, 0) u(\vec p, \hp) +
		 \sqrt{\frac{1}{3}} \varepsilon^{\mu}(\vec p, +1) u(\vec p, \hm),
		 \\
		 \psi^{\mu} (\vec p, -\frac{1}{2}) = &
		 \sqrt{\frac{2}{3}} \varepsilon^{\mu}(\vec p, 0) u(\vec p, \hm) +
		 \sqrt{\frac{1}{3}} \varepsilon^{\mu}(\vec p, -1) u(\vec p, \hp),
		 \\
		 \psi^{\mu} (\vec p, -\frac{3}{2}) = &
		 \varepsilon^{\mu}(\vec p, -1) u(\vec p, \hm),
	\end{align}
	which is example of addition of $1/2$ and $1$ spinor forms.
	Spinor $u(\vec p, \tau)$ can be taken same as for nucleon, except with
	the mass $M_\Delta = 1232 \text{MeV}$. Polarization vector $\varepsilon^{\mu}$
	is coming from representation for massive particle of spin $1$.
	We refer to appendix \ref{sec:polarization} for explicit formulas for this quantity
	and discussion about the way it is derived for arbitrary angle.
% 	\begin{align}
% 	T_{\frac{1}{2}+\frac{3}{2},\frac{1}{2}+\frac{1}{2}} = &
% 	-\frac{i e^{i \phi } \sin \theta  \left(2 M_\Delta M_N \left(s-M_\Delta^2\right)+s (1-\cos \theta)
% 	\left(s-M_N^2\right)\right)}{\sqrt{3} \sqrt{s} (1-\cos \theta)) \sqrt{\left(s-M_\Delta^2\right)
% 	\left(s-M_N^2\right)}} g_1(Q^2) \nonumber
% 	\\
% 	+ & \frac{-ie^{i \phi }M_\Delta-M_N\left(2 s \left(s-M_\Delta^2\right)-M_\Delta^2 \sin \theta  
% 	(1-\cos \theta) \left(s-M_N^2\right)\right.}{2 \sqrt{3} \sqrt{s} (1-\cos \theta) \sqrt{\left(s-M_\Delta^2\right)
% 	\left(s-M_N^2\right)}} g_2(Q^2) \nonumber
% 	\\
% 	+ & -\frac{i e^{i \phi } \sin \theta  (M_\Delta-M_N) \sqrt{\frac{\left(s-M_\Delta^2\right)
% 	\left(s-M_N^2\right)}{s}}}{\sqrt{3}} g_3(Q^2) \nonumber
% 	\end{align}

\section{$\Delta(1232)$ contribution $e N \to e N$ scattering amplitude}
	Previous diagram contributes to the reaction $eN\to e\Delta$ in the leading
	order of $\alpha_{em}$. One can consider contribution to another important
	reaction, $eN \to eN$, sketched on fig. \ref{fig:delta-box}. This contribution
	has order $O(\alpha^2)$, but nontheless can be topical for discussion of 
	discrepancies from leading order.
	
	\begin{figure} \centering
	\label{fig:delta-box}
	\begin{fmffile}{delta-box}
		\begin{fmfgraph*}(40,20)
			\fmfbottom{i1,d1,o1}
			\fmftop{i2,d2,o2}
			\fmf{fermion,label=$N(p',,\lambda')$}{v2,o1}
			\fmf{double,label=$\Delta$}{v1,v2}
			\fmfv{decor.shape=circle,decor.filled=hatched,
				decor.size=3thick}{v1,v2}
			\fmf{fermion,label=$N(p,,\lambda)$}{i1,v1}
			\fmf{fermion,label=$e(k,,h)$}{i2,v3}
			\fmf{fermion}{v3,v4}
			\fmf{fermion,label=$e(k',,h')$}{v4,o2}
			\fmf{photon,tension=0,label=$\gamma$}{v1,v3}
			\fmf{photon,tension=0,label=$\gamma$}{v2,v4}
		\end{fmfgraph*}
	\end{fmffile}
	\caption{Box-diagram contribution to reaction $eN \to eN$ with %
		$\Delta$ intermediate particle.}
	\end{figure}
	
	In general, 16 different amplitudes $T_{h'\lambda',h\lambda}$ 
	for different helicity states exist. However, due to conservation 
	laws some of them appear to be trivial:
	\begin{itemize}
		\item Assuming the mass of electron $m_e$ is relatively small 
			to energies of the problem, we can neglect it. In this 
			limit, helicity of the electron is conserved, so only half 
			of amplitudes survive ($16/2 = 8$);
		\item Parity is conserved in this channel with high accuracy, 
			therefore half of nonzero amplitudes appear to be equal 
			to another ones. Number of independent amplitudes is 
			smaller by factor of $2$ ($8/2 = 4$);
		\item Time-inversion symmetry connects two of remaining four 
			independent amplitudes.			
	\end{itemize}
	Therefore we are free to discuss only $3$ amplitudes: 
	$T_{1} = T_{\hp \hp, \hp \hp}, T_{2} = T_{\hp \hm, \hp \hp},$
	$T_{3} = T_{\hp \hm, \hp \hm}$.
	
	The most general form of non-flip amplitude $T$ is:
	\begin{equation} \label{eqn:str-amp} 
		T^{non-flip} = \frac{e^2}{Q^2} \bar{u}(k',h') \gamma_\mu u(k,h) 
		\, \times \, \bar{u}(p',\lambda') 
		\left(\cG_M  \gamma^\mu - \cF_2 
		\frac{P^{\mu}}{M} + \cF_3  \frac{\hat{K} P^{\mu}}{M^2} 
		\right) u(p,\lambda).
	\end{equation}
	
	One can write down amplitudes explicitly, considering two vertices
	from eqn. \ref{eqn:big-gamma}, four propagators, and different helicity
	spinors. Calculating products it is possible to write down equation
	that will connect form-factors $g_1, g_2, g_3$ from $\gamma N \to\Delta$ vertex
	and its contribution to lepton-nucleon scattering form-factors $G_M, F_2, F_3$.
	
	However, there is another way to tie up those quantities exist.
	Amplitudes of processes from fig. \ref{fig:delta-amp} and
	fig. \ref{fig:delta-box} can be connected by unitarity considerations
	known as optical theorem.
	
\section{Optical Theorem}
	In principle, time-evolution in quantum physics is unitary. This means, that S-matrix
	is unitary operator:
	\begin{equation}\label{eqn:unitarity-s}
		S^+ S = 1.
	\end{equation}
	For needs of quantum field theory as pertrubation theory from some "free" state,
	it is convenient to rewrite this operator in the following form:
	\begin{equation}
		S = 1 + i T.
	\end{equation}
	The last equation is written in interaction representation, so that in absent of
	interaction $T = 0$. All corrections, therefore any scattering is present
	due to nonzero $T$. In the terms of $T$ equation (\ref{eqn:unitarity-s}) can be
	rewritten in the form:
	\begin{equation}\label{eqn:unitarity-t}
		i (T^+ - T) =  T^+ T.
	\end{equation}
	We can use this equation to produce relationships for scattering amplitudes.
	For such purpose we surround it by state vectors. Say we have $2 \to 2$ scattering
	with momentum states $| k, p \rangle \to | k', p' \rangle$. On the left side,
	we get scattering amplitudes:
	\begin{equation} \label{eqn:pre-imaginary}
		-i[ T(in \to out) - T^\ast(out \to in)].
	\end{equation}
	On the right side, we can substract ''completeness relation`` to get the following:
	\begin{align}
		 \langle out | T^+ T | in \rangle & =
		 \sum_{\alpha}  \langle out | T^+| \alpha \rangle \langle \alpha |T | in \rangle 
		 \nonumber
		 \\
		 & = \sum_n \left(\prod_{i=1}^n \int \frac{d^3 q_i}{(2 \pi)^3} \frac{1}{3 q^0_i}
		 \right) T(in \to \{q_i\}) T^*(out \to \{q_i\}) (2 \pi)^4
		 \delta^4(k + p - \sum_i q_i)
	\end{align}
	In the case of forward scattering $| out \rangle = | in \rangle$, and we get
	optical theorem:
	\begin{equation}
		\Im T(k,p \to k,p) = \sum_i \int d\Pi_i | T(k,p \to \{q_i\}) |^2.
	\end{equation}
	For arbitrary scattering mode analogous relation can be not so neat.
	
	For the amplitudes (\ref{eqn:str-amp}) for elastic lepton-nucleon scattering,
	one can derive following formulas for three helicities:
	\begin{align}\label{eqn:form-factors}
		T_1 & = 2 \frac{e^2}{Q^2} \left\{ \frac{s u - M_N^4}{s - M_N^2}
		\left( \cF_2 - \cG_M - \frac{s - M_N^2}{2 M_N^2} \cF_3 \right) + Q^2 \cG_M
		\right\}
		\nonumber
		\\
		T_2 & = - \frac{e^2}{Q^2} \frac{\sqrt{Q^2(M_N^4 - s u)}}{M_N} \left\{
		\cF_2 + \frac{2 M_N^2}{s - M_N^2} (\cF_2 - \cG_M) - \cF_3
		\right\} e^{-i \phi}
		\nonumber
		\\
		T_3 & = 2 \frac{e^2}{Q^2} \frac{s u - M_N^4}{s - M_N^2}\left\{
		\cF_2 - \cG_M - \frac{s - M_N^2}{2 M_N^2} \cF_3 \right\}
	\end{align}
	For scattering to angle $\theta_{cm}$ one have $\phi = 0$, and instead of
	(\ref{eqn:pre-imaginary}) one can write simply $\Im T_1, \Im T_2, \Im T_3$
	respectively. Those quantities can be written by simply substituting $\Im \cF_i$
	instead of $\cF_i$ to right side of (\ref{eqn:form-factors}).
	
	Now we consider contribution discussed in previous section. Following above
	procedure, we can write down following equations:
	\begin{equation}\label{eqn:optical-diags}
	 2 \Im T_{h' \lambda', h \lambda}  =
	 \sum_{\chi, \tau} \int \frac{d^3 q_e}{(2 \pi)^3} \frac{1}{2 q^0_e}
	 \int \frac{d^3 q_\Delta}{(2 \pi)^3} \frac{1}{2 q^0_\Delta}
	 T^{*}_{\chi \tau , h' \lambda'}(k' p' \to q_e q_\Delta)
	 T_{\chi \tau, h \lambda}(k' p' \to q_e q_\Delta)
	 (2 \pi)^4 \delta^4 (p+k-q_e - q_\Delta).
	 \\ 
	\end{equation}
	We can use this equation to connect amplitudes of $eN \to eN$ and $eN \to e \Delta$
	in the way discussed in the end of the previous section. It is worth noting that
	if we insert diagram of order $O(\alpha_{em})$ on the right side (and its conjugate
	partner), we will obtain imaginary part of amplitude of order $O(\alpha_{em}^2)$
	on the left side. This is exactly what we need.
	
	Integration on the right side of (\ref{eqn:optical-diags}) can be simplified.
	In nucleon case integration can be easily simplified:
	\begin{align}
		 \int \frac{d^3 q_e d^3 q_\Delta}{(2 \pi)^6} \frac{1}{4 q^0_e q^0_\Delta}
		(2 \pi)^4 \delta^4 (k + p-\sum_i q_i)
		& =  \frac{1}{(2 \pi)^2 4 E_1 E_2} \int q_1^2 d q_1sin \theta d \theta d \phi
		\delta( |q_1| + \sqrt{q_1^2 + M_\Delta^2} - \sqrt{s} )
		\nonumber
		\\
		& =  \frac{1}{4 (2 \pi)^2} \int \frac{q_1^2}{q_1 (|q_1| + \sqrt{q_1^2 + M^2})}
		\delta( q_1 - q) d q_1sin \theta d \theta d \phi
		\nonumber
		\\
		& =  \frac{1}{16 \pi^2} \frac{q}{|q| + \sqrt{q^2 + M_\Delta^2}}
		\int sin \theta d \theta d \phi
		\nonumber
		\\
		& =  \frac{1}{32 \pi^2} \frac{\sqrt{\Lambda(s)}}{s} \int d \Omega \nonumber
	\end{align}
	with $ \Lambda(s) = ( s - M_\Delta^2 )^2 $.
	In this way we can rewrite set of equations (\ref{eqn:optical-diags})
	in a suitable form:
	\begin{align}
		 \Im T_1 & = \frac{\sqrt{\Lambda(s)}}{64 \pi^2 s} \int d \Omega
		 \left[ \tilde{T}_{\hp \tp, \hp \hp} T_{\hp \tp, \hp \hp}
		  + \tilde{T}_{\hp \hp, \hp \hp} T_{\hp \hp, \hp \hp}
		  + \tilde{T}_{\hp \hm, \hp \hp} T_{\hp \hm, \hp \hp}
		  + \tilde{T}_{\hp \tm, \hp \hp} T_{\hp \tm, \hp \hp}
		 \right],
		 \nonumber
		 \\
		 \Im T_2 & = \frac{\sqrt{\Lambda(s)}}{64 \pi^2 s} \int d \Omega
		 \left[ \tilde{T}_{\hp \tp, \hp \hm} T_{\hp \tp, \hp \hp}
		  + \tilde{T}_{\hp \hp, \hp \hm} T_{\hp \hp, \hp \hp}
		  + \tilde{T}_{\hp \hm, \hp \hm} T_{\hp \hm, \hp \hp}
		  + \tilde{T}_{\hp \tm, \hp \hm} T_{\hp \tm, \hp \hp}
		 \right],
		 \nonumber
		 \\
		 \Im T_3 & = \frac{\sqrt{\Lambda(s)}}{64 \pi^2 s} \int d \Omega
		 \left[ \tilde{T}_{\hp \tp, \hp \hm} T_{\hp \tp, \hp \hm}
		  + \tilde{T}_{\hp \hp, \hp \hm} T_{\hp \hp, \hp \hm}
		  + \tilde{T}_{\hp \hm, \hp \hm} T_{\hp \hm, \hp \hm}
		  + \tilde{T}_{\hp \tm, \hp \hm} T_{\hp \tm, \hp \hm}
		 \right], 
	\end{align}
	where by $\tilde{T_{....}}$ we mean both complex conjugation and using
	outcoming $eN$ states.
\begin{appendices}
\section{Massive Spin $S=1$ Polarization Vectors}\label{sec:polarization}

	For massive particle with spin 1 there are three independent polarizations.
	They are described by polarization vector $\epsilon^{\mu}(\vec{p}, \lambda)$,
	where $\lambda = 0$ stands for longitudinal polarization (absent in case
	of massless particle), $\lambda = \pm 1$ are right and left polarized 
	respectively.
	We will talk about polarization along  momentum $\vec{p}$, i.e. \textit{helicity}.
	Such convention is meaningless in the rest frame (for a massive particle),
	but in such case all three polarizations are equivalent. Anyway, we will
	mostly work in center of mass frame (with other particle present), so 
	below equations will be essential.

	General formula for polarization 4-vector can be written as follows:
	\begin{equation}
		\epsilon^{\mu}(\vec p, \lambda) = \left( \frac{\vec{p}\cdot 
		\vec{\varepsilon_{\lambda}}}{M} , \vec{\varepsilon_{\lambda}}
		+ \frac{\vec{p}\cdot 
		\vec{\varepsilon_{\lambda}}}{p^0 + M} \frac{\vec{p}}{M}
		\right),
	\end{equation}
	with $\vec{\varepsilon}_{\lambda}$ - some 3-vectors, which also called
	polarization vectors ($p^0 = \sqrt{\vec{p}^2 + M^2}$).
	For particle with momentum along $z$-axis, this vectors are straightforward:
	\begin{align} \label{eqn:polarization-3-vectors}
		\vec{\varepsilon}_{\lambda = 0}= & (0, 0 ,1), \\
		\vec{\varepsilon}_{\lambda = \pm 1}= & \mp \frac{1}{\sqrt{2}}(1, \pm i,0).
	\end{align}
	For general $\vec{p} = (p \sin\theta \cos\phi, p \sin\theta \sin\phi,
	p \cos(\phi)$ corresponding expressions are bit cumbersome.

	For \textbf{longitudinal} polarization, polarization is collinear with
	monentum vector, i.e. $\vec{\varepsilon}_{\lambda = 0} = \vec{p}/|\vec{p}|$.
	Substitution to general formula yields:
	\begin{equation}
		\epsilon^{\mu}(\vec p, \lambda = 0) = \left( \frac{|\vec{p}|}{M} ,
		\frac{p^0 \vec{p}}{M |\vec{p}|}	\right).
	\end{equation}
	To obtain \textbf{transversal} polarization, we will use another trick.
	Suppose we work in coordinate system, where $\vec{p} = p \hat{z}$.
	Then in this frame polarization 3-vectors are taken from 
	(\ref{eqn:polarization-3-vectors}). Now we apply rotational operators
	to this vectors, to get spinors, which satisfy following condition:
	\begin{equation}
		(\vec{J} \cdot \vec{n} ) \epsilon_{\lambda} = \lambda \epsilon_{\lambda}.
	\end{equation}
	Here $\vec{J} = (J_1, J_2, J_3)$ - spin operator, which in our representation
	has the form:
	\begin{equation}
		J_1 = \jx,\, J_2 = \jy,\, J_3 = \jz. 
	\end{equation}
	Then, we write (quite general) rotational operator:
	\begin{equation}
		R(\theta, \phi) = e^{-i \phi J_3} e^{- i \theta J_2} e^{+ i \phi J_3}.
	\end{equation}
	Taking matrix exponent is straightforward:
	\begin{equation*}
	e^{-i \phi J_3} = 
	\left(
	\begin{array}{ccc}
	\cos \phi & -\sin\phi & 0 \\
	\sin \phi & \cos \phi & 0 \\
	0 & 0 & 1 \\
	\end{array}
	\right),\,
	e^{-i \theta J_2} = 
	\left(
	\begin{array}{ccc}
	\cos \theta & 0 & \sin \theta \\
	0 & 1 & 0 \\
	-\sin \theta & 0 & \cos \theta \\
	\end{array}
	\right).
	\end{equation*}
	Using rotational operator on $\vec \varepsilon_{\lambda = 0}$ gives us
	previous result. For $\vec \varepsilon_{\lambda = \pm 1}$ one can get
	\begin{align} \label{eqn:polarization-3-vectors}
		\vec{\varepsilon}_{\lambda = +1}= & -\frac{1}{\sqrt{2}} &
		\left(\cos^2 \frac{\theta}{2} - e^{+2 i \phi}\sin^2 \frac{\theta}{2} ,
		+i(\cos^2 \frac{\theta}{2} + e^{+2 i \phi}\sin^2 \frac{\theta}{2}) ,
		- e^{+i \phi} sin \theta \right),
		\\
		\vec{\varepsilon}_{\lambda = -1}= & +\frac{1}{\sqrt{2}} &
		\left(\cos^2 \frac{\theta}{2} - e^{-2 i \phi}\sin^2 \frac{\theta}{2} , 
		-i(\cos^2 \frac{\theta}{2} + e^{-2 i \phi}\sin^2 \frac{\theta}{2}) ,
		- e^{-i \phi} sin \theta \right),
	\end{align}
	Substituting these 3-vectors to general formula for 4-vector, one gets:
	\begin{equation}
		\epsilon^{\mu}(\vec p, \lambda = \pm 1) = \left( 0 ,
		\vec{\varepsilon}_{\lambda = \pm 1} \right).
	\end{equation}

	Notice, that $p_{\mu} \epsilon^{\mu} = 0$ for $p$ on mass shell.
	
\section{Amplitudes for $eN -> e\Delta$ scattering}\label{sec:in-to-intermediate}	
	
\end{appendices}
	
\begin{thebibliography}{99}
	\bibitem{bib:leader}
		E.~ Leader, {\it Spin in particle physics\/}, Cambridge University Press,
		Cambridge (2001).
\end{thebibliography}

\end{document}