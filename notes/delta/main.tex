\documentclass{article}
\usepackage{feynmp}
\usepackage{amsmath,amssymb,amsthm,amsbsy,color,paralist}

\newcommand{\hp}{{\frac{1}{2}}}
\newcommand{\hm}{{-\frac{1}{2}}}

\newcommand{\bk}{{\boldsymbol k}}
\newcommand{\bq}{{\boldsymbol q}}
\newcommand{\bp}{{\boldsymbol p}}
\newcommand{\bK}{{\boldsymbol K}}
\newcommand{\bP}{{\boldsymbol P}}

\newcommand{\cG}{{\cal G}}
\newcommand{\cF}{{\cal F}}
\newcommand{\cma}{{\theta_{cm}}}
\newcommand{\cM}{{\cal M}}

\begin{document}
	\unitlength = 1mm
	% determine the unit for the size of diagram.
\section{ $eN \to e\Delta$ reaction}
	Under certain circumstances, reaction $eN \to e\Delta$ reaction
	can occur. In lowest order, amplitude can be discribed by the
	tree-level diagram (fig. \ref{fig:delta-amp}).
	\begin{figure} \centering \label{fig:delta-amp}
	\begin{fmffile}{delta-amp}
		\begin{fmfgraph*}(40,20)
			\fmfbottom{i1,d1,o1}
			\fmftop{i2,d2,o2}
			\fmf{double,label=$\Delta(\mu,,\tau)$}{v1,o1}
			\fmfv{decor.shape=circle,decor.filled=hatched,
				decor.size=3thick}{v1}
			\fmf{fermion,label=$N(p,,\lambda)$}{i1,v1}
			\fmf{fermion,label=$e(k,,h)$}{i2,v2}
			\fmf{fermion,label=$e(k',,h')$}{v2,o2}
			\fmf{photon,tension=0,label=$\gamma$}{v1,v2}
		\end{fmfgraph*}
	\end{fmffile}
	\caption{Tree-level contribution to reaction $eN \to e\Delta$.}
	\end{figure}
	
	Analytically this amplitude can be expressed in terms of photon propagator,
	fermionic spinors and vertices:
	\begin{equation}
		T = \frac{e^2}{-q^2} \-{u}(k', h') \gamma_\mu u(k, h) \cdot
		\bar{\psi}_\alpha (p', \tau) \Gamma^{\alpha \mu} u(p, \lambda),
	\end{equation}
	where $\psi_\alpha (p', \tau)$ - spin $3/2$ particle spinor for $\Delta$
	(also known as Rarita-Schwinger spinor). $\gamma N \Delta$ vertice there
	can be described as following:
	
	\begin{align} \label{eqn:polarization-3-vectors}
	\Gamma^{\alpha \mu} = \sqrt{\frac{2}{3}}& \{ (\gamma^\mu 
	q^\alpha - \hat{q} g^{\alpha \mu})g_1(Q^2) 
	\\
	& + (q \cdot p' g^{\alpha \mu} - q^\alpha p^{\prime \mu}) g_2(Q^2)
	\\
	& + (q^\alpha q^\mu - q^2 g^{\alpha \mu}) g_3(Q^2)\} i \gamma_5,
	\end{align}

	\begin{equation}
		(\hat{p} - M_\Delta) \psi_\alpha = 0, \,\,\,\, \alpha = 0,1,2,3
	\end{equation}
	
	\begin{equation}
		p_\alpha \psi_\alpha(p, \tau) = 0.
	\end{equation}

	For different incoming and outcoming particles, one can construct
	$2 \times 2 \times 2 \times 4 = 32$ amplitudes $T_{h' \tau, h \lambda}$.
	However, in a limit of zero electron mass amplitudes with lepton 
	spin flip are zero. Moreover, parity conservation suggests following
	relation:
	\begin{equation}
		T_{h' \tau, h \lambda} = T_{-h' -\tau, -h -\lambda}
	\end{equation}
	Therefore, only $8$ independent amplitudes should be considered.
	
\section{$\Delta(1232)$ contribution $e N \to e N$ scattering amplitude}
	Previous reaction
	\begin{figure} \centering
	\begin{fmffile}{delta-box}
		\begin{fmfgraph*}(40,20)
			\fmfbottom{i1,d1,o1}
			\fmftop{i2,d2,o2}
			\fmf{fermion,label=$N(p',,\lambda')$}{v2,o1}
			\fmf{double,label=$\Delta$}{v1,v2}
			\fmfv{decor.shape=circle,decor.filled=hatched,
				decor.size=3thick}{v1,v2}
			\fmf{fermion,label=$N(p,,\lambda)$}{i1,v1}
			\fmf{fermion,label=$e(k,,h)$}{i2,v3}
			\fmf{fermion}{v3,v4}
			\fmf{fermion,label=$e(k',,h')$}{v4,o2}
			\fmf{photon,tension=0,label=$\gamma$}{v1,v3}
			\fmf{photon,tension=0,label=$\gamma$}{v2,v4}
		\end{fmfgraph*}
	\end{fmffile}
	\caption{Box-diagram contribution to reaction $eN \to eN$ with %
		$\Delta$ intermediate particle.}
	\end{figure}
	
	In general, 16 different amplitudes $T_{h'\lambda',h\lambda}$ 
	for different helicity states exist. However, due to conservation 
	laws some of them appear to be trivial:
	\begin{itemize}
		\item Assuming the mass of electron $m_e$ is relatively small 
			to energies of the problem, we can neglect it. In this 
			limit, helicity of the electron is conserved, so only half 
			of amplitudes survive ($16/2 = 8$);
		\item Parity is conserved in this channel with high accuracy, 
			therefore half of nonzero amplitudes appear to be equal 
			to another ones. Number of independent amplitudes is 
			smaller by factor of $2$ ($8/2 = 4$);
		\item Time-inversion symmetry connects two of remaining four 
			independent amplitudes.			
	\end{itemize}
	Therefore we are free to discuss only $3$ amplitudes: 
	$T_{1} = T_{\hp \hp, \hp \hp}, T_{2} = T_{\hp \hm, \hp \hp},$
	$T_{3} = T_{\hp \hm, \hp \hm}$.
	
	The most general form of non-flip amplitude $T$ is:
	\begin{equation} \label{str_amp} 
		T^{non-flip} = \frac{e^2}{Q^2} \bar{u}(k',h') \gamma_\mu u(k,h) 
		\, \times \, \bar{u}(p',\lambda') 
		\left(\cG_M  \gamma^\mu - \cF_2 
		\frac{P^{\mu}}{M} + \cF_3  \frac{\hat{K} P^{\mu}}{M^2} 
		\right) u(p,\lambda).
	\end{equation}
	
\end{document}